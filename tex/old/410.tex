\documentclass[notes]{subfiles}
\begin{document}
	\addcontentsline{toc}{section}{4.10 - Antiderivatives}
	\setcounter{section}{9}
	\refstepcounter{section}
	\fancyhead[RO,LE]{\bfseries \nameref{cs410}} 
	\fancyhead[LO,RE]{\bfseries \small \currentname}
	\fancyfoot[C]{{}}
	\fancyfoot[LO,RE]{\large \thepage}	%Footer on Right \thepage is pagenumber
	\fancyfoot[RO,LE]{\large Chapter 4.10}
	
\section*{Antiderivatives}\label{cs410}
	\subsection*{Before Class}
	\subsubsection*{Derivatives}
		Before we begin with antiderivatives, here is a table of derivatives we have so far:
			\begin{center}
				\setlength\arrayrulewidth{1.5pt}
				\tabulinesep = 4mm
				\begin{tabu} {| X[.75,c] | X[c] || X[.75,c] | X[c] |}\hline
					\multicolumn{4}{|c|}{\large \bfseries Table of Derivatives}\\ \hline
					\textbf{Function}	& \textbf{Derivative}	& \textbf{Function}	& \textbf{Derivative} \\ \hline
					$c$				& $0$					& $x^n$				& $nx^{n-1}$ \\ \hline
					$\sin x$			& $\cos x$				& $\cos x$			& $-\sin x$ \\ \hline
					$\tan x$			& $\sec^2x$				& $\cot x$			& $-\csc^2x$ \\ \hline
					$\sec x$			& $\sec x\tan x$			& $\csc x$			& $-\csc x\cot x$\\ \hline
					$e^x$			& $e^x$				& $b^x$			& $\ln b\cdot b^x$ \\ \hline
					$\inv{\sin}(x)$		& $\dfrac{1}{\sqrt{1-x^2}}$ & $\inv{\cos}(x)$	& $-\dfrac{1}{\sqrt{1-x^2}}$ \\ \hline
					$\inv{\tan}(x)$		& $\dfrac{1}{1 + x^2}$	& $\inv{\cot}(x)$	& $-\dfrac{1}{1+x^2}$ \\ \hline
					$\inv{\sec}(x)$		& $\dfrac{1}{x\sqrt{x^2-1}}$ & $\inv{\csc}(x)$	& $-\dfrac{1}{x\sqrt{x^2-1}}$ \\ \hline
					$\ln x$			& $\dfrac{1}{x}$		& $\log_b x$		& $\dfrac{1}{\ln b}\cdot \dfrac{1}{x}$ \\ \hline
					$c\cdot f(x)$		& $c\cdot f'(x)$			& $f(x)\pm g(x)$		& $f'(x)\pm g'(x)$\\ \hline
					$f(x)\cdot g(x)$	& $f'(x)g(x) + f(x)g'(x)$	& $\dfrac{f(x)}{g(x)}$& $\dfrac{f'(x)g(x)-f(x)g'(x)}{[g(x)]^2}$\\ \hline
					$f(g(x))$		& $f'(g(x))\cdot g'(x)$		& 					& \\ \hline
				\end{tabu}
			\end{center}
		\newpage
		
	\subsubsection*{The Antiderivative}
		\begin{defn}[Antiderivative]
			A function \(F\) is called an \textbf{antiderivative} of \(f\) on an interval \(I\) if 
				\blank{2}\\ \\\blank{2}.
		\end{defn}
		
		\begin{ex}
			Find five antiderivatives for the function \(f(x) = 2x\).
		\end{ex}
			\vs{1}
			
		\begin{question}
			How might we write the general antiderivative \(F(x)\) for \(f(x) = 2x\)?  
		\end{question}
			\vs{1}
			
		\begin{ex}
			Find the most general antiderivative for the functions \(f(x) = \cos x\), \(g(x) = x^n\) (\(n\geq 0\)), and \(h(x) = x^{-2}\).
		\end{ex}
			\vs{2}
			\newpage
			
		A table of useful antiderivatives is given below:
			\begin{center}
				\setlength\arrayrulewidth{1.5pt}
				\tabulinesep = 4mm
				\begin{tabu} {| X[.75,c] | X[c] || X[.75,c] | X[c] |}\hline
					\multicolumn{4}{|c|}{\large \bfseries Table of Antiderivatives}\\ \hline
					\textbf{Function}	& \textbf{Antiderivative}	& \textbf{Function}	& \textbf{Antiderivative} \\ \hline
					$0$				& $0$					& $k$				& $kx$ \\ \hline
					$x^n$			& $\dfrac{x^{n+1}}{n+1}$		& $\sin x$			& $-\cos x$ \\ \hline
					$\cos x$			& $\sin x$				& $\sec^2 x$			& $\tan x$ \\ \hline
					$\csc^2 x$			& $-\cot x$			& $\sec x\tan x$		& $\sec x$\\ \hline
					$\cot x\csc x$		& $-\csc x$			& $e^x$		& $e^x$\\ \hline
					$b^x$			& $\dfrac{1}{\ln b}\cdot b^x$ & $\dfrac{1}{\sqrt{1-x^2}}$	& $\inv{\sin}(x)$ \\ \hline
					$\dfrac{1}{1+x^2}$	& $\inv{\tan}(x)$		& $\dfrac{1}{x\sqrt{x^2-1}}$ & $\inv{\sec}(x)$ \\ \hline
					$f(x)\pm g(x)$		& $F(x)\pm G(x)$		& & \\ \hline
				\end{tabu}
			\end{center}
		
		\begin{ex}
			\(g'(x) = 3\sin x - \dfrac{9x^4-\sqrt[3]{x}}{x^2}\).  Find \(g(x)\), the most general antiderivative of \(g\).
		\end{ex}
			\vs{1}
			
		\begin{ex}
			Find \(f\) if \(f'(x) = x\sqrt{x}\) and \(f(1) = 5\).
		\end{ex}
			\vs{1}
			\newpage
			
	\subsection*{Pre Class Practice}
		\begin{ex}
			Find the most general antiderivative of the following functions; check your answer by differentiation.
			\begin{enumerate}[(a)]
				\item \(f(x) = 5x + 3\)
					\vs{1}
					
				\item \(f(x) = x(12x + 8)\)
					\vs{1}
					
				\item \(g(x) = \sqrt{5}\)
					\vs{1}
					
				\item \(h(x) = -3\sqrt{x} - 8\sqrt[3]{x}\)
					\vs{1}
					
				\item \(k(x) = \dfrac{6}{x^7}\)
					\vs{1}
					
				\item \(f(t) = \dfrac{5-4t^3 + 2t^6}{t^6}\)
					\vs{1}
			\end{enumerate}
		\end{ex}
			\newpage
		
	\subsection*{In Class}	
	\subsubsection*{Examples}
		\begin{ex}
			Find \(f\) if \(f''(x) = 12x^2+6x-\sin(x)\), \(f(0) = 4\), and \(f(1) = 2\).
		\end{ex}
			\vs{1}
			
		\begin{ex}
			A particle moves in a straight line, and its acceleration is given by \(a(t) = 3t+2\).  Its initial velocity is \(v(0) = -3\) cm/s, and its initial displacement is \(s(0) = 5\) cm.  Find the position function \(s(t)\).
		\end{ex}
			\vs{1}
			\newpage
			
		\begin{ex}
			A ball is thrown upward with speed 24 ft/s from the edge of a cliff which is 432 ft above the ground.  Find its height above the ground \(t\) seconds later.  When does it reach its maximum height?  When does it hit the ground?
		\end{ex}
			\vs{2}
			
		\begin{defn}[Indefinite Integral]
			The \textbf{indefinite integral} of \(f\) is a family of functions \(F(x)\) such that
			\blank{1.4}$ $\\[25pt] or \blank{3}
		\end{defn}
		
		\begin{ex}
			Write the indefinite integral for \(\ds \int x^2\,dx\).  
		\end{ex}
			\vs{1}
			\newpage
			
		\begin{rmk}[Properties of Indefinite Integrals]
			Let \(F\) and \(G\) be the antiderivatives of \(f\) and \(g\), respectively, and let \(k\) be a real number. Then,\\[10pt]
			\begin{itemize}
				\setlength\itemsep{15pt}
				\item \(\ds \int (f(x)\pm g(x))\, dx = \)
				\item \(\ds \int kf(x)\, dx = \)
			\end{itemize}
		\end{rmk}
		
		\begin{ex}
			Find the indefinite integral of \(\ds \int \dfrac{4}{1+x^2}\, dx\) 
		\end{ex}
			\vs{1}
			
		\begin{ex}
			Find \(\ds \int f(x)\, dx\) for the functions below.
			\begin{enumerate}[(a)]
				\item \(f(x) = e^x -3x^2 + \sin x\)
					\vs{1}
					
				\item \(f(x) = \dfrac{1}{\sqrt{x}}\)
					\vs{1}
					\newpage
					
				\item \(f(x) = x^{1/3} + (2x)^{1/3}\)
					\vs{1}
					
				\item \(f(x) = \sec x\tan x + 4x\)
					\vs{1}
					
				\item \(f(x) = 4\sqrt{x} + \sqrt[4]{x}\)
					\vs{1}
			\end{enumerate}
		\end{ex}
		
		\begin{ex}
			Solve the initial value problems:
			\begin{enumerate}[(a)]
				\item \(f'(x) = x^{-3}\), \(f(1) =1 \)
					\vs{1}
					
				\item \(f'(x) = x^3-8x^2 + 16x + 1\),  \(f(0) = 0\)
					\vs{1}
			\end{enumerate}
		\end{ex}
			\newpage
			
	\subsection*{After Class Practice}
		\begin{ex}
			Determine the antiderivative/indefinite integral of the following functions.
			\begin{enumerate}[(a)]
				\item \(f(x) = 5x^4 + 4x^5\)
					\vs{1}
					
				\item \(g(x) = x-1+4\sin(2x)\)
					\vs{1}
					
				\item \(h(x) = \dfrac{14x^3+2x+1}{x^3}\)
					\vs{1}
					
				\item \(k(x) = 8\sec x(\sec x -4\tan x)\)
					\vs{1}
					
				\item \(j(x) = \sin x\cos x\)
					\vs{1}
			\end{enumerate}
		\end{ex}
			\newpage
			
		\begin{ex}
			A car is being driven at a rate of 40 mph when the brakes are applied. The car decelerates at a constant rate of 10 ft/sec\(^2\). How long does it take for the car to stop?
		\end{ex}	
			\vs{1}
			
		\begin{ex}
			Solve the initial value problem \(f''(x) = e^{-x}\), \(f'(0) = -2\), \(f(0) = 5\)
		\end{ex}
			\vs{1}
			
		\begin{ex}
			The following is a false statement: If \(f(x)\) is the antiderivative of \(v(x)\), then \(f(2x)\) is the antiderivative of \(v(2x)\). Find an explict example of \(f(x)\) and \(v(x)\) and argue why the statement is false.
		\end{ex}
			\vs{1}
			
		
			
		
\clearpage
\end{document}
