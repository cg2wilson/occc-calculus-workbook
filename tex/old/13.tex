\documentclass[notes]{subfiles}
\begin{document}
	\addcontentsline{toc}{section}{1.3 - Trigonometric Functions}
	\setcounter{section}{3}
	\fancyhead[RO,LE]{\bfseries \large \nameref{cs13}} 
	\fancyhead[LO,RE]{\bfseries \currentname}
	\fancyfoot[C]{{}}
	\fancyfoot[LO,RE]{\large \thepage}	%Footer on Right \thepage is pagenumber
	\fancyfoot[RO,LE]{\large Chapter 1.3}

\section*{Trigonometric Functions}\label{cs13}
	\subsection*{Definitions/Properties}
		\begin{center}
			\setlength{\arrayrulewidth}{1.5pt}
			\renewcommand{\arraystretch}{2}	
	 		\begin{tabular}{| ccc | c |} \hline
	 			\multicolumn{4}{|c|}{\large{\textbf{Basic Trigonometric Functions}}}\\ \hline
	 			\(\sin x = \dfrac{\text{opp}}{\text{hyp}}\) & \(\cos x = \dfrac{\text{adj}}{\text{hyp}}\) & \(\tan x = \dfrac{\text{opp}}{\text{adj}} \)	& \\
	 			&&&
	 			\multirow{3}{*}{
	 				\begin{tikzpicture}[scale = .9]
 						\coordinate (one) at (0,0);
 						\coordinate (two) at (4,3); 
 						\coordinate (three) at (4,0);
						\draw (one)--(two) node[midway, yshift = 9pt, sloped] {hyp};
						\draw (two)--(three) node[midway, xshift = 14pt, yshift = -6pt] {opp};
						\draw (three)--(one) node[midway, yshift = -9pt] {adj};
						\draw (.75,0) node[yshift = 5pt] {$x$};
					\end{tikzpicture}
					} \\ & & & \\
				$\csc x = \dfrac{\text{hyp}}{\text{opp}}$ & $\sec x = \dfrac{\text{hyp}}{\text{adj}}$ & $\cot x = \dfrac{\text{adj}}{\text{opp}}$ & \\
				&&& \\ \hline
	 		\end{tabular}
		\end{center}
		
		\begin{rmk}[IMPORTANT NOTE]
			Trigonometric functions are \emph{functions}, and always require an input (an angle). In particular, an expression like
				\[\cos (x) = \dfrac{1}{3}\]
			is appropriate, while 
				\[\cos = \dfrac{1}{3}\]
			is \textbf{inappropriate}.
		\end{rmk}
		
		\begin{center}
				\setlength{\arrayrulewidth}{1.5pt}
				\renewcommand{\arraystretch}{2}
				\begin{tabular}{| c | c | c | c | c | c | c | } \cline{2-7}
					\multicolumn{1}{c|}{} & \multicolumn{6}{c|}{{\large \textbf{Properties of Trig Functions}}}\\ \cline{2-7}
					\multicolumn{1}{c|}{} & \(\sin x\) 		& \(\cos x\) 			& \(\tan x\) 											& \(\cot x\) 							& \(\sec x\) 											  & \(\csc x\) \\ \hline
					& &&&&&\\[-20pt]
					\textbf{Domain}		& \((-\infty,\infty)\)& \((-\infty,\infty)$	& \makecell{\(x\neq \dfrac{\pi}{2} + \pi k\) \\ \(k\in \Z\)}	& \makecell{\(x\neq \pi k\)\\ \(k\in \Z\)} 	& \makecell{\(x\neq \dfrac{\pi}{2} + \pi k\) \\ \(k\in \Z\)} & \makecell{\(x\neq \pi k\)\\ \(k\in \Z\)} \\ \hline
					\textbf{Range}		& $[-1,1]$		& $[-1,1]$ 			& $(-\infty,\infty)$									& $(-\infty,\infty)$					& \makecell{$(-\infty,-1]\cup $\\ $[1,\infty)$}							  &  \makecell{$(-\infty,-1]\cup $\\ $[1,\infty)$} \\ \hline
					\textbf{Period}& $2\pi$ & $2\pi$ & $\pi$ & $\pi$ & $2\pi$ & $2\pi$ \\ \hline
				\end{tabular}
			\end{center}
			\newpage
	\subsection*{Radian Measure}
		\begin{defn}[Radians]
			A \textbf{radian} is defined to be the angle required to make the length of an arc on the circle the same length as the radius of the circle.
		\end{defn}
		\begin{rmk}[Converting Between Degrees and Radians]
			\qquad \qquad \qquad \(\pi\) radians \( = 180\dc\)  \qquad	\(1\dc = \dfrac{\pi}{180}\) radians \qquad 1 radian \(= \dfrac{180\dc}{\pi}\)
		\end{rmk}
		
		\begin{ex}
			Convert \(\dfrac{\pi}{45}\) radians into degrees, and convert \(215\dc\) into radians.
		\end{ex}
			\vs{1}

	\subsection*{Arc Length and Sector Area}
		\begin{rmk}[Arc Length]
			The length of an arc \(s\) of a circle of radius \(r\) is given by
				\[s = r\theta\]
			where \(\theta\) is the inscribed angle \emph{in radians}
		\end{rmk}
		
		\begin{rmk}[Area of a Sector]
			The area \(A\) of a sector of a circle of radius \(r\) is given by
				\[A = \dfrac{1}{2}r^2\theta\]
			where \(\theta\) is the inscribed angle \emph{in radians}
		\end{rmk}
			\newpage
			
		\begin{ex}
			Find the area for a sector of a circle of radius 4 with an inscribed angle of \(60\dc\).
		\end{ex}
			\vs{1}
			
	\subsection*{Unit Circle}
		Knowing the values of the unit circle will make your life much easier; fill it out below.

			\begin{tikzpicture}[scale=5.3,cap=round,>=latex]
			        % draw the coordinates
			        \draw[->] (-1.5cm,0cm) -- (1.5cm,0cm) node[right,fill=white] {\(x\)};
			        \draw[->] (0cm,-1.5cm) -- (0cm,1.5cm) node[above,fill=white] {\(y\)};
			
			        % draw the unit circle
			        \draw[thick] (0cm,0cm) circle(1cm);
			
			        \foreach \x in {0,30,...,360} {
			                % lines from center to point
			                \draw[gray] (0cm,0cm) -- (\x:1cm);
			                % dots at each point
			                \filldraw[black] (\x:1cm) circle(0.4pt);
			                % draw each angle in degrees
					\fill[white] (\x:0.4cm) circle (0.12);
			        }
			        
			        \foreach \x in {45,135,225,315} {
			        	% lines from center to point
			                \draw[gray] (0cm,0cm) -- (\x:1cm);
			                % dots at each point
			                \filldraw[black] (\x:1cm) circle(0.4pt);
			                % draw each angle in degrees
					\fill[white] (\x:0.4cm) circle (0.12);
			        }
			
			        % draw each angle in radians
			        \foreach \x/\xtext in {
			            30/\frac{\pi}{6},
			            45/\frac{\pi}{4},
			            60/\frac{\pi}{3},
			            90/\frac{\pi}{2},
			            120/\frac{2\pi}{3},
			            135/\frac{3\pi}{4},
			            150/\frac{5\pi}{6},
			            180/\pi,
			            210/\frac{7\pi}{6},
			            225/\frac{5\pi}{4},
			            240/\frac{4\pi}{3},
			            270/\frac{3\pi}{2},
			            300/\frac{5\pi}{3},
			            315/\frac{7\pi}{4},
			            330/\frac{11\pi}{6},
			            360/2\pi}
			                \fill[white] (\x:0.8cm) circle (0.12);

				\foreach \x/\xtext/\y in {
			            % the coordinates for the first quadrant
	    			    30/\frac{\sqrt{3}}{2}/\frac{1}{2},
			     		45/\frac{\sqrt{2}}{2}/\frac{\sqrt{2}}{2},
			            60/\frac{1}{2}/\frac{\sqrt{3}}{2},
			            % the coordinates for the second quadrant
			            150/-\frac{\sqrt{3}}{2}/\frac{1}{2},
			            135/-\frac{\sqrt{2}}{2}/\frac{\sqrt{2}}{2},
			            120/-\frac{1}{2}/\frac{\sqrt{3}}{2},
			            % the coordinates for the third quadrant
			            210/-\frac{\sqrt{3}}{2}/-\frac{1}{2},
			            225/-\frac{\sqrt{2}}{2}/-\frac{\sqrt{2}}{2},
			            240/-\frac{1}{2}/-\frac{\sqrt{3}}{2},
			            % the coordinates for the fourth quadrant
			            330/\frac{\sqrt{3}}{2}/-\frac{1}{2},
			            315/\frac{\sqrt{2}}{2}/-\frac{\sqrt{2}}{2},
			            300/\frac{1}{2}/-\frac{\sqrt{3}}{2}}
			                \draw (\x:1.25cm) node[fill=white] {};
			
			        % draw the horizontal and vertical coordinates
			        % the placement is better this way
			        \draw (-1.25cm,0cm) node[above=1pt] {$(-1,0)$}
			              (1.25cm,0cm)  node[above=1pt] {$(1,0)$}
			              (0cm,-1.25cm) node[fill=white] {$(0,-1)$}
			              (0cm,1.25cm)  node[fill=white] {$(0,1)$};
	   	 \end{tikzpicture}
		\newpage
	\subsection*{Identities}
		\begin{center}
			\setlength{\arrayrulewidth}{1.5pt}
			\renewcommand{\arraystretch}{1.5}
			\begin{tabular}{|c|c|} \hline
				{\large \textbf{Quotient Identities}} & {\large \textbf{Reciprocal Identities}}\\ \hline
				& \\[-15pt]
				\(\tan x = \dfrac{\sin x}{\cos x}\) & \(\csc x = \dfrac{1}{\sin x}\) \\[15pt]
				\(\cot x = \dfrac{\cos x}{\sin x}\) & \(\sec x = \dfrac{1}{\cos x}\) \\[15pt]
										& \(\cot x = \dfrac{1}{\tan x}\) \\[15pt] \hline
			\end{tabular}
		\end{center}
			
		\begin{center}
			\setlength{\arrayrulewidth}{1.5pt}
			\renewcommand{\arraystretch}{1.5}
			\begin{tabular}{|c|c|} \hline
				{\large \textbf{Double Angle Identities}} & {\large \textbf{Pythagorean Identities}} \\ \hline
				& \\[-15pt]
				\(\sin (2x) = 2\sin x\cos x\) & \(\sin^2 x + \cos^2 x =1 \) \\ 
				\(\cos (2x) = 2\cos^2x - 1\) & \(1 + \cot^2 x = \csc^2x\) \\
				\(\cos (2x) = 1-2\sin^2x\) & \(\tan^2 x + 1 = \sec^2 x\) \\
				\(\cos (2x) = \cos^2x-\sin^2x\) & \\ \hline
			\end{tabular}
		\end{center}
	\subsection*{Trig Equations}
		\begin{rmk}[Solving Trig Equations]
			Solving equations involving trigonometric functions is similar in process to solving polynomial functions. Three things must be kept in mind:
			\begin{itemize}
				\item A trigonometric function \textbf{must} be of the form: \(\text{trig}\,(x)\)
				\item The range of the function in question.
				\item The interval(s) involved in the question.
			\end{itemize}
		\end{rmk}
			\newpage
			
		\begin{ex}
			Solve the equation \(\cos^2x - 2\cos x + 1 = 0\) on the interval \([0,4\pi)\).
		\end{ex}
			\vs{1}
			
		\begin{ex}
			Solve the equation \(\sin (3x) = \dfrac{1}{2}\) on the interval \([0,2\pi)\) and in general.
		\end{ex}
			\vs{1}
		
\clearpage
\end{document}