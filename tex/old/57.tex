\documentclass[notes]{subfiles}

\begin{document}
	\addcontentsline{toc}{section}{5.7 - Integrals Resulting in Inverse Trig Functions}
	\refstepcounter{section}
	\fancyhead[RO,LE]{\bfseries \large\nameref{cs57}} 
	\fancyhead[LO,RE]{\bfseries \currentname}
	\fancyfoot[C]{{}}
	\fancyfoot[RO,LE]{\large \thepage}	%Footer on Right \thepage is pagenumber
	\fancyfoot[LO,RE]{\large Chapter 5.7}
	
\section*{Integrals Resulting in Inverse Trig Functions}\label{cs57}
	\subsection*{The Rules}
	
	There are three major functions which give inverse trig functions:
	
	\begin{rmk}[Integrals Resulting in Inverse Trig Functions]
		Let \(a\) be a constant. Then, \\[125pt]
	\end{rmk}
	
	\begin{ex}
		Evaluate \(\ds \int_0^{1/2} \dfrac{dx}{\sqrt{1-x^2}}\, dx\)
	\end{ex}	
		\vs{1}
		\newpage
		
	\begin{ex}
		Evaluate the integral \(\ds \int \dfrac{1}{\sqrt{4-9x^2}}\, dx\)
	\end{ex}
		\vs{1}
		
	\begin{ex}
		Find \(\ds \int \dfrac{1}{16 + x^2}\, dx\)
	\end{ex}
		\vs{1}
		
	\begin{ex}
		Evaluate \(\ds \int_0^2 \dfrac{dx}{8+2x^2}\, dx\)
	\end{ex}
		\vs{1}
	
	
	\clearpage			
\end{document}