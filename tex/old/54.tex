\documentclass[notes]{subfiles}

\begin{document}
	\addcontentsline{toc}{section}{5.4 - Integration Formulas \& the Net Change Theorem}
	\refstepcounter{section}
	\fancyhead[RO,LE]{\bfseries \large\nameref{cs54}} 
	\fancyhead[LO,RE]{\bfseries \currentname}
	\fancyfoot[C]{{}}
	\fancyfoot[RO,LE]{\large \thepage}	%Footer on Right \thepage is pagenumber
	\fancyfoot[LO,RE]{\large Chapter 5.4}
	
\section*{Integration Formulas \& the Net Change Theorem}\label{cs54}
	\subsection*{Integration Formulas}	
			\tabulinesep = 3mm
			{\setlength{\arrayrulewidth}{1.5pt}
			\begin{tabu}{| X[l] X[l] | X[l] X[l] | }\hline
				\multicolumn{4}{|c|}{\large{\textbf{Integration Formulas}}} \\ \hline
				\(\ds \int c\cdot f(x)\, dx\) 		& \(c\ds \int f(x)\, dx\)			& \(\ds \int [f(x)\pm g(x)]\, dx\) 		& \(\ds \int f(x)\, dx \pm \int g(x)\, dx\) \\
				\(\ds \int k\, dx\)				& \(kx + C\)					& \(\ds \int x^n\, dx\)					& \(\dfrac{x^{n+1}}{n+1} + C\), \(n\neq 1\)\\
				\(\ds \int \sin x\, dx\)			& \(-\cos x + C\)				& \(\ds \int \cos x\, dx\)				& \(\sin x + C\)\\
				\(\ds \int \sec^2x\, dx\)		& \(\tan x + C\)				& \(\ds \int \csc^2x\, dx\)				& \(-\cot x + C\)\\
				\(\ds \int \sec x\tan x\, dx\)		& \(\sec x + C\)				& \(\ds \int \csc x\cot x\, dx\)			& \(-\csc x + C\)\\ 
				\(\ds \int \dfrac{1}{\sqrt{1-x^2}}\, dx\)	& \(\inv{\sin}(x) +C\)	& \(\ds \int \dfrac{1}{1+x^2}\,dx\)		& \(\inv{\tan}(x) + C\)\\
				\(\ds \int \dfrac{1}{x\sqrt{x^2-1}}\, dx\)	& \(\inv{\sec}(x) + C\)& \(\ds \int e^x\, dx\)				& \(e^x + C\) \\
				\(\ds \int b^x\, dx\)			& \(\dfrac{1}{\ln b} b^x + C\)	& \(\ds \int \dfrac{1}{x}\, dx\)			& \(\ln |x| + C\)\\ \hline
			\end{tabu}
			}

			\newpage
			
		\begin{ex}
			Find the general indefinite integral for \(f(x) = 3x^5-2\csc^2 x\)
		\end{ex}
			\vs{1}
			
		\begin{ex}
			Evaluate \(\ds \int \dfrac{\sin\theta}{\cos^2\theta}\,d\theta\)
		\end{ex}
			\vs{1}
			
		\begin{ex}
			Evaluate \(\ds \int (6e^x-2\cos x)\, dx\)
		\end{ex}
			\vs{1}
			\newpage
			
		\begin{ex}
			Compute the following:
			\begin{enumerate}[(a)]
				\item  \(\ds \int \lrpar{\cos x + \dfrac{1}{3}x}\, dx\)
					\vs{1}
				\item \(\ds \int \lrpar{1-x^2}^2\, dx\)
					\vs{1} 
				\item \(\ds \int_1^2 \lrpar{\dfrac{4x^3 - 3x^2 + 2x -3}{x}}\, dx\)
					\vs{1} 
			\end{enumerate}
		\end{ex}
	
		\begin{rmk}[Integrals of Symmetric Functions]
			For symmetric functions, we have the following properties:\\

			\showto{st}{
				\begin{enumerate}[(1)]
				\setlength\itemsep{5pt}
					\item If \(f\) is continuous on \([-a,a]\) and \(f\) is even, then\vspace{.75in}
					\item If \(f\) is continuous on \([-a,a]\) and \(f\) is odd, then\vspace{.75in}
				\end{enumerate}
			}
		\end{rmk}
			\newpage
			
	\subsection*{The Net Change Theorem}
		\begin{question}
			\begin{enumerate}[(a)]
				\item In \S5.1, how did we find the accumulated change of a function?  
					\vs{1}
					
				\item In \S5.3, we learned the Fundamental Theorem of Calculus.  Rewrite FTC 2 here.
					\vs{1}
			\end{enumerate}
		\end{question}
		
		\begin{thm}[Net Change]
			The integral of a rate of change is the net change, i.e.:
				\\ \\ \\ 
		\end{thm}
		\begin{question}
			What relationship(s) do you see between the Net Change Theorem and FTC 2?
		\end{question}
			\vs{1}
			
		\begin{rmk}[Displacement/Distance]
			When talking about physical situations, the \emph{displacement} of a particle is the
				\blank{1} \\ \\ \blank{3}, while the \emph{distance} is the 
				\blank{1.5} \\ \\ \blank{3.5}.
		\end{rmk}
			\newpage
			
		\begin{ex}
			A particle moves along a line so that its velocity at time \(t\) is \(v(t) = t^2-t-6\) m/s. 
			\begin{enumerate}[(a)]
				\item Find the displacement of the particle during the time period \(1\leq t\leq 4\).
					\vs{1}
					
				\item Find the distance traveled during this time period.
					\vs{1}
					
			\end{enumerate}
		\end{ex}
		
		\begin{ex}
			A particle moving along a line has velocity \(v(t) = t^2 -2t-3\) m/s.  Find the displacement and the total distance traveled by the particle between 1 and 4 seconds.
		\end{ex}
			\vs{1.5}
			\newpage
			
	\subsection*{Examples}
		\begin{ex}
			Find the general indefinite integral of \(f(x) = x^{1.3}-7x^{2.5}\)
		\end{ex}
			\vs{1}
			
		\begin{ex}
			Find the general indefinite integral of \(f(x) = \sqrt[5]{x^4}\)
		\end{ex}
			\vs{1}
			
		\begin{ex}
			Find the general indefinite integral of \(f(x) = \dfrac{1-\sqrt{x}+x}{\sqrt{x}}\)
		\end{ex}
			\vs{1}
			
		\begin{ex}
			Find the general indefinite integral of \(f(x) = 2+\tan^2x\)
		\end{ex}
			\vs{1}
			\newpage
			
		\begin{ex}
			Find the general indefinite integral of \(f(t) = \dfrac{5}{1+x^2}\)
		\end{ex}
			\vs{1}
			
		\begin{ex}
			Evaluate the integral \(\ds \int_{-2}^3 (x^2-3)\, dx\)
		\end{ex}
			\vs{1}
			
		\begin{ex}
			Evaluate the integral \(\ds \int_0^3 (1 + 6w^2-10w^4)\,dw\)
		\end{ex}
			\vs{1}
			
		\begin{ex}
			Evaluate the integral \(\ds \int_0^\pi (4\sin \theta -3\cos\theta)\, d\theta\)
		\end{ex}
			\vs{1}
			
		\begin{ex}
			Evaluate the integral \(\ds \int_0^{\pi/3} \dfrac{\sin\theta + \sin\theta\tan^2\theta}{\sec^2\theta}\,d\theta\)
		\end{ex}
			\vs{1}
			\newpage
			
		\begin{ex}
			Evaluate the integral \(\ds \int_1^8 \dfrac{2+t}{\sqrt[3]{t^2}}\)
		\end{ex}
			\vs{1}
			
		\begin{ex}
			A honeybee population starts with 100 bees and increases at a rate of \(n'(t)\) bees per week.  What does 100 + \(\ds \int_0^{15}n'(t)\,dt\) represent?
		\end{ex}
			\vs{1}
			
		\begin{ex}
			If \(x\) is measured in meters and \(f(x)\) is measured in newtons, what are the units of \(\ds \int_0^{100}f(x)\,dx\)?
		\end{ex}
			\vs{1}
			
		\begin{ex}
			The acceleration function of a particle is \(a(t) = t+4\) m/s\(^2\), and its the initial velocity is 5 m/s.  Find the velocity at time \(t\), and the distance traveled between time \(t = 0\) and \(t = 5\).
		\end{ex}
			\vs{1}
			\newpage
			
		\begin{ex}
			Evaluate \(\ds \int_{-\pi/3}^{\pi/3} x^4\sin x\,dx\)
		\end{ex}
			\vs{1}
			
		\begin{ex}
			Compute \(\ds \int_{-\pi/4}^{\pi/4} \lrpar{x^3 + x^4 \tan x}\, dx\)
		\end{ex}
			\vs{1}
			
		\begin{ex}
			Compute \(\ds \int_{-\pi/6}^{\pi/6} \dfrac{1}{\sqrt{1-x^2}}\, dx\)
		\end{ex}
			\vs{1}
			
	
			
	\clearpage			
\end{document}