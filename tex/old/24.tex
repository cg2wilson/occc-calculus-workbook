\documentclass[notes]{subfiles}
\begin{document}
	\addcontentsline{toc}{section}{2.4 - Continuity}
	\refstepcounter{section}
	\fancyhead[RO,LE]{\bfseries \nameref{cs24}} 
	\fancyhead[LO,RE]{\bfseries \small \currentname}
	\fancyfoot[C]{{}}
	\fancyfoot[LO,RE]{\large \thepage}	%Footer on Right \thepage is pagenumber
	\fancyfoot[RO,LE]{\large Chapter 2.4}
	
\section*{Continuity}\label{cs24}
	\subsection*{Before Class}
	\subsubsection*{The Definition}
		\begin{defn}[Continuity]
			A function is \textbf{continuous at input} \(a\) if \\[50pt]

			 If this condition fails, then \(f\) is said to be \blank{2.5}. \(f\) is \textbf{continuous} \\[20pt] \textbf{on an interval} if it is \blank{4.5}.
		\end{defn}
			\vs{.25}
			
		\begin{rmk}[Conditions for Continuity]
			Continuity of a function requires three things:\\ \\
				\begin{enumerate}
				\setlength\itemsep{15pt}
					\item \blank{4} 
					\item \blank{4}
					\item \blank{4}
				\end{enumerate}

		\end{rmk}
			\vs{.25}
		
		\begin{ex}
			For some function \(f(x)\), we know that \(\ds \lim_{x\to 5} f(x)\) exists. Is it necessarily true that the function is continuous at \(x=5\)? Why or why not?
		\end{ex}
			\vs{1}
			\newpage
			
		\begin{ex}
		Use the graph to find the following:
		\begin{center}
			\begin{tikzpicture}
				\begin{axis}[
					scale = 1.5,
					every tick label/.append style={font=\small},
					axis x line = middle,
					axis y line = middle,
		    			every axis y label/.style={at={(ticklabel cs:1.15)}},
		    			ytick = {-4, -2, -3, -1, 1, 2, 3, 4},
					y label style={at={(axis description cs:.5,1.15)},anchor=north},
		    			ylabel = {$f(x)$},
		    			ymin = -5, ymax = 5,
	    				every axis x label/.style= {at ={(ticklabel cs:1)}},
	    				xtick = {-4,-3,-2,-1,1,2,3,4},
	    				x label style={at={(axis description cs:1.1,.5)},anchor=east},
	    				xlabel = {$x$},
	    				xmin = -5, xmax = 5			
					]
					\addplot[<-,thick,smooth,domain = 1.4:3.] {4/(x-1)-5};
					\addplot[->,thick,smooth,domain = 3.:4.9] {x-6};
					\addplot[<-,thick,smooth, domain = -4.9:1]  {2^x -4};
					\coordinate (circle1) at (1,-2);
					\coordinate (circle2) at (3,-3);
					\coordinate (circle3) at (3,2);
				\end{axis}
					\fill[white] (circle1) circle (.1);
					\draw[thick] (circle1) circle (.1);
					\fill[white] (circle2) circle (.1);
					\draw[thick] (circle2) circle (.1);
					\fill (circle3) circle (.1);
			\end{tikzpicture}
		\end{center}\vspace*{20pt}
		\begin{minipage}{\textwidth}
			\begin{multicols*}{3}
			\begin{enumerate}[(a)]
				\item $\ds \lim_{x\to 1^+} f(x)$\\[50pt]
				\item $\ds \lim_{x\to 1^-} f(x)$\\[50pt] 
				\item $\ds \lim_{x\to 1} f(x)$\\[50pt] 
				\item Is $f$ continuous at $x =1 $?\\[50pt]
					\columnbreak
				\item $\ds \lim_{x\to 3^+} f(x)$\\[50pt] 
				\item $\ds \lim_{x\to 3^-} f(x)$\\[50pt]  
				\item $\ds \lim_{x\to 3} f(x)$\\[50pt] 
				\item Is $f$ continuous at $x =3 $?\\[50pt]
					\columnbreak
				\item $\ds \lim_{x\to 0^+} f(x)$\\[50pt] 
				\item $\ds \lim_{x\to 0^-} f(x)$\\[50pt]  
				\item $\ds \lim_{x\to 0} f(x)$\\[50pt] 
				\item Is $f$ continuous at $x =0 $?\\[50pt]
			\end{enumerate}
				\raggedcolumns
			\end{multicols*}
		\end{minipage}
		\end{ex}
			\newpage
			
		\begin{rmk}[Types of Discontinuities]
			There are three kinds of discontinuities which we will encounter:\\[10pt]
			\begin{itemize}
			\setlength\itemsep{15pt}
				\item \blank{3}
				\item \blank{3}
				\item \blank{3}
			\end{itemize}
		\end{rmk}
		
		\begin{ex}
			Sketch an example of each of the three kinds of discontinuities.
		\end{ex}
			\vs{1}
			
		\begin{defn}[Continuity from the Right and Left]
			A function \(f\) is said to be \textbf{continuous from the right} at input \(a\) if

				\\ \\ \\ \\
			
			and \(f\) is \textbf{continuous from the left} at input \(a\) if 

				\\ \\
			
		\end{defn}	
			\newpage
			
	\subsection*{Pre Class Practice}
		\begin{ex}
			Use the graph below to answer the following questions:\\
			\begin{minipage}{3in}
				\includegraphics[scale = .58]{2.4fig1}
			\end{minipage}
			\begin{minipage}{3.8in}
				\begin{enumerate}[(a)]
					\item State the numbers at which \(f\) is discontinuous, and classify each as an \emph{infinite discontinuity}, \emph{jump discontinuity}, or \emph{removable discontinuity}.
					\item For each of the numbers in (a), determine whether \(f\) is right-continuous, left-continuous, or neither.  
					\item Use interval notation to write where \(f\) is continuous.
				\end{enumerate}
			\end{minipage}
		\end{ex}
			\vs{1}
			
		\begin{ex}
			Sketch the graph of a function \(f\) that is continuous everywhere, but is:
				\begin{itemize}
					\item Discontinuous at \(x = 5\)
					\item Left-continuous at \(x = -1\)
					\item Has a removable discontinuity at \(x = 3\)
				\end{itemize}
		\end{ex}
			\vs{1}	
			\newpage
			
	\subsection*{In Class}
	\subsubsection*{Useful Results}
		\begin{thm}[Composite Function Theorem]
			If \(f(x)\) is continuous at \(L\) and \(\ds \lim_{x\to a} g(x) = L\), then \\[50pt]
		\end{thm}
			
		\begin{ex}
			Evaluate \(\ds \lim_{x\to \pi} \sin (x-\pi)\)
		\end{ex}
			\vs{1}
			
		\begin{thm}[Properties of Continuous Functions]
			If \(f\) and \(g\) are continuous at \(a\), and if \(c\) is a constant, then the following functions are all continuous at \(a\): \\ \\
				\begin{enumerate}
				\setlength\itemsep{15pt}
					\item \blank{3}
					\item \blank{3}
					\item \blank{3}
					\item \blank{3}
					\item \blank{3}
				\end{enumerate}
		\end{thm}
			\newpage
			
		\begin{ex}
			Prove the first property of continuous functions.
		\end{ex}
			\vs{1}
		
		\begin{thm}[Types of Continuous Functions]
			The following functions are continuous on their domain: \\ \\
				\begin{itemize}
				\setlength\itemsep{15pt}
					\item \blank{3}
					\item \blank{3}
					\item \blank{3}
					\item \blank{3}
					\item \blank{3}
				\end{itemize}
			
		\end{thm}	
			\vs{.25}
			\newpage
			
		\begin{ex}
			Find the interval(s) on which the following functions are continuous
			\begin{enumerate}[(a)]
				\item \(f(x) = x^{67}-47.521x^{51}+719\)
					\vs{1}
					
				\item \(g(x) = \dfrac{x^2+2x+17}{x^2-1}\)
					\vs{1}
					
				\item \(h(x) = \sqrt{x} + \dfrac{x+1}{x-1} + \dfrac{x+1}{x^2+1}\)
					\vs{1}
					
				\item \(\cos\lrpar{\dfrac{1}{x}}\)
					\vs{1}
					
				\item \(\ln (x^2 - 25)\)
					\vs{1}					
			\end{enumerate}
		\end{ex}
			\newpage
			
	\subsection*{The Intermediate Value Theorem}
		\begin{thm}[Intermediate Value Theorem]
			Suppose that $f$ is continuous on the closed interval $[a,b]$, and let $N$ be any number between $f(a)$ and $f(b)$, where $f(a)\neq f(b)$.  Then, \\ \\ \\
		\end{thm}
		Here is a sketch illustrating why the Intermediate Value Theorem is true:
			\vs{1}	

		\begin{question}
			Why does a function need to be continuous in order to guarantee the conclusion of the Intermediate Value Theorem?  Sketch a graph to support your idea.
		\end{question}
			\vs{1}
			\newpage
			
		\begin{ex}
			Show that the equation \(4x^3-6x^2+3x-2 = 0\) has a root between 1 and 2.
		\end{ex}
			\vs{1}
			
		\begin{ex}
			For what value of the constant \(c\) is the function \(f\) continuous on \((-\infty,\infty)\)?
			\[f(x) = \begin{cases} cx^2+2x & x < 2 \\ x^3-cx & x\geq 2\end{cases}\]
		\end{ex}
			\vs{1}
			\newpage
	\subsection*{After Class Practice}
		\begin{ex}
			Suppose \(f\) and \(g\) are continuous function such that \(g(2) = 6\) and \(\ds \lim_{x\to 2} [3f(x) + f(x)g(x)] = 36\).  What is \(f(2)\)?
		\end{ex}
			\vs{1}
			
		\begin{ex}
			Find the numbers at which \(g\) is discontinuous.  At which of these numbers is \(f\) right-continuous, left-continuous, or neither?
				\[ g(x) = \begin{cases}x^2 + 1 & x \leq 1\\ 3-x & 1 < x \leq 4 \\ \sqrt{x} & x > 4 \end{cases}\]
		\end{ex}
			\vs{1}
			\newpage
			
		\begin{ex}
			Find values of \(a\) and \(b\) that make \(f\) continuous everywhere.
				\[f(x) = \begin{cases}\dfrac{x^2-4}{x-2} & x < 2 \\ ax^2 - bx + 3 & 2\leq x < 3 \\ 2x - a + b & x \geq 3 \end{cases}\]
		\end{ex}
			\vs{1}
		
		\begin{ex}
			A function \(f\) is a ratio of quadratic functions and has a vertical asymptote \(x=4\) and just one \(x-\)intercept, \(x=1\).  We know that \(f\) has a removable discontinuity at \(x=-1\), and that \(\ds \lim_{x\to -1} f(x) = 2\).  Evaluate \(f(0)\) and find any horizontal asymptotes of \(f\).
		\end{ex}
			\vs{1}

\clearpage
\end{document}
