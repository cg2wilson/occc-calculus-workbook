\documentclass[notes]{subfiles}
\begin{document}
	\chapter{Algebra \& Trig Review}
	\addcontentsline{toc}{section}{1.1 - Review of Functions}
	\setcounter{section}{1}
	\fancyhead[RO,LE]{\bfseries \large \nameref{cs11}} 
	\fancyhead[LO,RE]{\bfseries \currentname}
	\fancyfoot[C]{{}}
	\fancyfoot[LO,RE]{\large \thepage}	%Footer on Right \thepage is pagenumber
	\fancyfoot[RO,LE]{\large Chapter 1.1}

\section*{Functions}\label{cs11}
	
	\subsection*{Basics}
		\begin{defn}[Relation/Domain/Range]
			A \textbf{relation} is a collection of ordered pairs of the form \((x,y)\). The collection of first components in a relation is called the \textbf{domain} of the relation. The collection of second components in a relation is called the \textbf{range} of the relation.
		\end{defn}
		
		\begin{defn}[Input/Output]
			In an ordered pair \((x,y)\), the element \(x\) is called an \textbf{input} and the element \(y\) is called an \textbf{output}. 
		\end{defn}
		
		\begin{defn}[Function]
			A \textbf{function} \(f\) is a rule which assigns to each input \(x\) in the domain a single output \(f(x)\) in the range.
		\end{defn}
		
		\begin{ex}
			The relation \(f\) is copied below. Is \(f\) a function? Why or why not?
			\[\lrbrace{(-2,1), (1,5), (0,-3), (-2,3), (10,1), (3,3)}\]
		\end{ex}
			\vs{1}
			\newpage
			
		\begin{defn}[Graph]
			The \textbf{graph} of a function is the collection of all coordinate pairs \((x,y)\), plotted on a coordinate system.
		\end{defn}
			\vs{1}
		
		\begin{thm}[Vertical Line Test]
			A relation is a function if and only if any vertical intersects the graph of the relation in \emph{at most} one point.
		\end{thm}
	
		\begin{ex}
			Are both of these graphs functions?  Why or why not?
			\begin{center}
				\begin{tabular}{lr}
					\begin{tikzpicture}
						\begin{axis}[
						axis x line = middle,
	    						axis y line = middle,
	    					every axis y label/.style={at={(ticklabel cs:1.1)}},
							y label style={at={(axis description cs:0,1.1)},anchor=north},
	    					ylabel = {$y$},
	    						every axis x label/.style= {at ={(ticklabel cs:1)}},
	    						x label style={at={(axis description cs:1.1,.48)},anchor=east},
	    						xlabel = {$x$},
						trig format plots = rad,
						xmin = 0, xmax = 6.5,
						ymin = -1.5, ymax = 1.5
						]
						\addplot[thick, smooth, domain = 0:2*pi] {cos(x)};
						\end{axis}
					\end{tikzpicture}
				&
					\begin{tikzpicture}
	  	  				\begin{axis}[
	    						axis x line = middle,
	    						axis y line = middle,
	    					every axis y label/.style={at={(ticklabel cs:1.1)}},
							y label style={at={(axis description cs:.5,1.1)},anchor=north},
	    					ylabel = {$y$},
	    						every axis x label/.style= {at ={(ticklabel cs:1)}},
	    						x label style={at={(axis description cs:1.1,.65)},anchor=east},
	    						xlabel = {$x$},
							xmin=-4.5,xmax=4.5,
	       			    		ymin=-9.5,ymax=4.5,
			       		    xtick = {-4,-2,2,4},
	    		   		    ytick = {-8,-6,-4,-2,2,4},
	        		    	]
	        			    \addplot [domain=-3:3,samples=50]({x^3-3*x},{3*x^2-9}); 
	   				 \end{axis}
					\end{tikzpicture}
				\end{tabular}
			\end{center}
		\end{ex}
			\vs{1}
	
		\begin{ex} 
			Below are numerical expressions for the relations $h$ and $k$.  Is $h$ a function?  What about $k$?
			\begin{center}
				{\renewcommand{\arraystretch}{1.2}
				\begin{tabular}{|c|c|c|c|c|c|c|} \hline
					$x$ & 0 & 1 & 1 & 2 & 5 & 6 \\ \hline
					$h(x)$ & 0& 1 & 2 & 3 & 4 & 5 \\ \hline
				\end{tabular}\hspace*{15pt}
				\begin{tabular}{|c|c|c|c|c|c|c|} \hline
						$t$ & 0 & 1 & 1 & 2 & 5 & 6 \\ \hline
						$k(t)$ & 0 & 1 & 1 & 3 & 4 & 5 \\ \hline
					\end{tabular}
				}
				\end{center}
		\end{ex}	
			\vs{1}
			\newpage
			
	\subsection*{Using Functions}
		\begin{rmk}[Function Notation: Functions of a Single Variable]
			The notation \(f(x)\) has two pieces: input and rule. We are applying the \emph{rule} we call \(f\) to the \emph{input} we call \(x\). The result is an output, usually denoted \(y\). So we can write
			\[y = f(x)\]
			as a way to say ``the output \(y\) is the result of applying the rule \(f\) to the input \(x\)''. You want to read this as ``\(y\) equals \(f\) of \(x\)''.
		\end{rmk}

		\begin{rmk}[Evaluating Functions]
			To evaluate a function \(f(x)\), replace all instances of the variable \(x\) with the input shown.
		\end{rmk}

		\begin{ex}
			Let \(f(x) = x^2 + 5\). Evaluate the following:
			\begin{multicols}{2}
				\begin{enumerate}[(a)]
					\item \(f(1)\)
						\vspace*{35pt}

					\item \(f(0)\)
						\vspace*{35pt}

					\item \(f(-2)\)
						\vspace*{35pt}

					\item \(f(a)\)
						\vspace*{35pt}

					\item \(f(x + h)\)
						\vspace*{35pt}
				\end{enumerate}
			\end{multicols}
		\end{ex}

		\begin{ex}
			Consider the following function rules: \(f(x) = x-1, g(x) = -x^2, h(x) = 2^x\). Write the function notation for the descriptions given, then compute.
			\begin{enumerate}[(a)]
				\item \(f\) applied to the input 5
					\vs{1}
					\newpage

				\item \(g\) applied to the input \(a + 1\)
					\vs{1}

				\item \(h\) applied to the input \(3 + r\)
					\vs{1}
			\end{enumerate}
		\end{ex}	
		
	\subsection*{Combining Functions}
		Functions can be combined using the operations addition, subtraction, multiplication, and division. The notation is given below.
		\begin{center}
			\begin{tabular}{cccc}
				\textbf{Name} & \textbf{Notation} & \textbf{Translation} & \textbf{Note}\\
				\emph{Sum} & \((f+g)(x)\) & \(f(x) + g(x)\) & \\
				\emph{Difference} & \((f-g)(x)\) & \(f(x) - g(x)\) & \\
				\emph{Product} & \((f\cdot g)(x)\) & \(f(x)\cdot g(x)\) & \(f\cdot g\) can be written as \(fg\) as well \\
				\emph{Quotient} & \(\lrpar{\dfrac{f}{g}}(x)\) & \(\dfrac{f(x)}{g(x)}\) & \(g(x)\neq 0\)
			\end{tabular}
		\end{center}	
		
		\begin{ex}
			Let \(f(x) = 3x - 1\) and \(g(x) = x^2 + x +1\). Write and simplify the following:
			\begin{enumerate}[(a)]
				\item \((f+g)(x)\)
					\vs{1}
					
				\item \((f-g)(x)\)
					\vs{1}
					
				\item \((fg)(x)\)
					\vs{1}
					
				\item \(\lrpar{\dfrac{f}{g}}(x)\)
					\vs{1}
			\end{enumerate}
		\end{ex}
			\newpage
		
		Functions can also be combined using \emph{composition}: using one function as an input to another. This notation is given by
			\[(f\circ g)(x) = f(g(x))\]	
		\begin{ex}
			Let \(f(x) = 3x - 1\) and \(g(x) = x^2 + x +1\). Write and simplify the following:
			\begin{enumerate}[(a)]
				\item \((f\circ g)(x)\)
					\vs{1}
					
				\item \((g\circ f)(x)\)
					\vs{1}
					
				\item \(g(-4x)\)
					\vs{1}
					
				\item \(f(g(-x))\)
					\vs{1}
			\end{enumerate}
		\end{ex}
		
		\begin{rmk}[Domain/Range of a Composition]
			Given two functions \(f(x)\) and \(g(x)\), the composition \((f\circ g)(x)\) has the domain
				\[\text{Dom}\,(f\circ g) = \lrbrace{x\mid x\in \text{Dom}\,(g) \text{ and } g(x)\in \text{Dom}\, (f)}\]
			and range
				\[\text{Range}\, (f\circ g) = \lrbrace{y = f(x) \mid x\in \text{Range}\, (g) }\]
		\end{rmk}
		
		\begin{ex}
			Consider the functions \(f\) and \(g\) given below.
			\begin{center}
				\begin{tabular}{|c|c|c|c|c|c|c|c|c|}\hline
					\(x\) & \(-3\) & \(-2\) & \(-1\) & \(0\) & \(1\) & \(2\) & \(3\) & \(4\) \\ \hline
					\(f(x)\) & \(0\) & \(4\) & \(2\) &\(4\) & \(-2\) & \(0\) & \(-2\) & \(4\) \\ \hline
				\end{tabular}
				\begin{tabular}{|c|c|c|c|c|c|}\hline
					\(x\) & \(-4\) & \(-2\) & \(0\) & \(2\) & \(4\) \\ \hline
					\(g(x)\) & \(1\) & \(0\) & \(3\) & \(0\) & \(5\) \\ \hline
				\end{tabular}
			\end{center}
			Find the following:
			\begin{enumerate}[(a)]
				\item \((g\circ f)(3)\)
					\vs{1}
					
				\item The domain and range of \((g\circ f)(x)\)
					\vs{1}
					
				\item The domain and range of \((f\circ g)(x)\)
					\vs{1}
			\end{enumerate}
		\end{ex}
			\newpage
			
	\subsection*{Symmetry}
		\begin{defn}[Even/Odd Function]
			A function \(f(x)\) is said to be \textbf{even} if it has the property that
				\[f(-x) = f(x)\]
			for all \(x\) in its domain. A function is said to be \textbf{odd} if it has the property that
				\[f(-x) = -f(x)\]
			for all \(x\) in its domain.
		\end{defn}
		
		\begin{ex}
			Determine if the following functions ar even, odd, or neither.
			\begin{enumerate}[(a)]
				\item \(f(x) = x^7 + x^5 - x\)
					\vs{1}
					
				\item \(g(x) = 3-x^2\)
					\vs{1}
					
				\item \(h(x) = x^2 - x^3\)
					\vs{1}
			\end{enumerate}
		\end{ex}
			
\clearpage
\end{document}

		\begin{defn}[Piecewise Function]
			A \textbf{piecewise function} is a function defined by different formulas in different parts of their domains.
		\end{defn}
						
		\begin{ex}
			A quick example of a piecewise function is the \emph{absolute value function}: 
				\[f(x) = |x| = \begin{cases}-x & x < 0 \\x & x \geq 0  \end{cases}\]
				
			\begin{enumerate}[(a)]
				\item What is \(f(-5)\)? What about \(f(1)\)?
					\vs{.5}
					
				\item What is \(f(0)\)?  Why?
					\vs{.5}
					
				\item Sketch \(|x|\) on the interval \(-5\leq x \leq 5\).
					\vs{1}
			\end{enumerate}
		\end{ex}
			\newpage
		
		\begin{ex}
			A function \(h\) is defined by \(h(x) = \begin{cases}3-x & x< 2\\ x^2+x & x \geq 2 \end{cases}\)
			\begin{enumerate}[(a)]
				\item Evaluate \(h(-2)\), \(h(3)\), and \(h(2)\).
					\vs{1}

					
				\item Sketch the graph of\(h\)
					\vs{1}
					
			\end{enumerate}
		\end{ex}
			
		\begin{ex}
			Write the absolute value function \(f(x) = |2x-3|\) as a piecewise function
		\end{ex}
			\vs{1}
			\newpage
			
		\begin{rmk}[Function Composition]
			Given two function \(f\) and \(g\), the \textbf{composite function} \(f\circ g\) is defined by \\[20pt]
			The domain of \(f\circ g\) is \blank{5}.
		\end{rmk}
		
		\begin{ex}
			For each function below, (1) find the domain of the composite function, (2) completely decompose the function into smaller ones.
			\begin{enumerate}[(a)]
				\item \(f(x) = \dfrac{1}{x+2}\)
					\vs{1}

				\item  \(q(x) = (2x+1)^5\)	
					\vs{1}
						
				\item  \(s(h) = \sin \left(5h^2 + \dfrac{1}{h}\right)\)		
					\vs{1}
					
				\item  \(y(r) = \dfrac{5.317}{(2r^5 + 1.7)^2}\)		
					\vs{1}
			\end{enumerate}
		\end{ex}
			\newpage
			
		\begin{ex}
			If \(f(x) = \sqrt{x}\) and \(g(x) = \sqrt{2-x}\), find and simplify
				\begin{enumerate}[(a)]
					\item \(f\circ g\)
						\vs{1}
						
					\item \(g\circ f\)
						\vs{1}
						
					\item \(f\circ f\)
						\vs{1}
						
					\item \(g\circ g\)
						\vs{1}
				\end{enumerate}
		\end{ex}
			
		\begin{ex}
			Let \(k(x) = \sec(x^2)\tan(x^2)\).  Find \(f,g\) such that \(k(x) = f(g(x))\).
		\end{ex}
			\vs{1}
			\newpage
			
		\begin{ex}
			Let \(f(x) = \cos^2 (x^2 + 9)\).  Find functions \(a,b,c\) such that \(f(x) = (a\circ b\circ c)(x)\)
		\end{ex}
			\vs{1}
		
		\begin{ex}
			Given \(f(x) = x^2 + x -1\) and \(g(x) = 2-x\), what is the equation of \(y =(f\circ g)(x)\)?
		\end{ex}
			\vs{1}
			
	\subsection*{Types of Functions}	
		\begin{defn}[Polynomial]
			A function \(P\) is called a \textbf{polynomial} if \\ \vspace{.5in}
				where \(n\) is \blank{3} and the numbers \(a_i\) are constants, called \\ \vspace{15pt} \blank{2}.  The \emph{degree} of the polynomial is  \blank{2} \\ \vspace{15pt} \blank {3.5}.
		\end{defn}
		
		\begin{flushleft}
	 		\tabulinesep = 2mm	 	
	 		\setlength{\arrayrulewidth}{1.5pt}	
	 		\begin{tabu}{| X[.5,c] | X[1,c] | X[c] | X[c] |}\hline 
	 			\multicolumn{4}{|c|}{{\large \textbf{Common Polynomials}}} \\ \hline
	 			
	 			\textbf{Polynomial} 	& \textbf{Equation}	& \textbf{Graph} (\(a_n > 0\))	& \textbf{Graph} (\(a_n < 0\)) \\ \hline 
	 								& & & \\
				Linear				& & &  \\ 
									& & &  \\
									& & & \\ \hline
									& & & \\
				Quadratic			& & &  \\ 
									& & &  \\
		 							& & & \\ \hline
		 							& & & \\
		 		Cubic				& & & \\ 
		 							& & &  \\
		 							& & & \\ \hline
		 							& & & \\
		 		Even Degree			& & & \\ 
		 							& & &  \\
		 							& & & \\ \hline
		 							& & & \\
		 		Odd Degree			& & & \\ 
		 							& & &  \\
		 							& & & \\ \hline
		 	\end{tabu}
	 	\end{flushleft}
	 	
		\begin{flushleft}
			\tabulinesep = 3mm
			\setlength{\arrayrulewidth}{1.5pt}	
	 		\begin{tabu}{| X[1,c]  X[1,c]  X[1,c] | X[1.5,c,p] |} \hline
	 			\multicolumn{4}{|c|}{\large{\textbf{Basic Trigonometric Functions}}}\\ \hline
	 			&&&\multirow{5}{*}{
	 				\begin{tikzpicture}[scale = .9]
 						\coordinate (one) at (0,0);
 						\coordinate (two) at (4,3); 
 						\coordinate (three) at (4,0);
						\draw (one)--(two) node[midway, yshift = 9pt, sloped] {hyp};
						\draw (two)--(three) node[midway, xshift = 14pt, yshift = -6pt] {opp};
						\draw (three)--(one) node[midway, yshift = -9pt] {adj};
						\draw (.75,0) node[yshift = 5pt] {\(x\)};
					\end{tikzpicture}
					} \\
	 			\(\sin x = \hspace*{20pt}\) & \(\cos x = \hspace*{20pt}\) & \(\tan x = \hspace*{30pt} \)	& \\
	 			 & & & \\
				\(\csc x = \hspace*{20pt}\) & \(\sec x = \hspace*{20pt}\)& \(\cot x = \hspace*{30pt}\) & \\ 
				&&& \\ \hline
	 		\end{tabu}
		\end{flushleft}
			\vs{1}
			
		Here are some useful properties of trigonometric functions:
		\begin{flushleft}
				\tabulinesep = 2mm
				\setlength{\arrayrulewidth}{1.5pt}
				\begin{tabu}{| X[.78,c] | X[c] | X[c] | X[1.2,c] | X[c] | X[1.4,c] | X[1.4,c] | } \cline{2-7}
					\multicolumn{1}{c|}{} & \(\sin x\) 		& \(\cos x\) 			& \(\tan x\) 											& \(\cot x\)							& \(\sec x\)											  & \(\csc x\) \\ \hline
										&  &  	&  	&   	&   &  \\
					\textbf{Domain}		&  &  	&  	&   	&   &  \\ 
										&  &  	&  	&   	&   &  \\ \hline
										&  &  	&  	&   	&   &  \\
					\textbf{Range}		&  & 	& 	& 	&   &  \\ 
										&  &  	&  	&   	&   &  \\ \hline
										&  &  	&  	&   	&   &  \\
					\textbf{Period}		&  & 	& 	& 	&   &  \\ 
										&  &  	&  	&   	&   &  \\ \hline
				\end{tabu}
			\end{flushleft}
				\vs{1}
			\newpage
			
		Knowing the values of the unit circle will make your life much easier; fill it out below.

			\begin{tikzpicture}[scale=5.3,cap=round,>=latex]
			        % draw the coordinates
			        \draw[->] (-1.5cm,0cm) -- (1.5cm,0cm) node[right,fill=white] {\(x\)};
			        \draw[->] (0cm,-1.5cm) -- (0cm,1.5cm) node[above,fill=white] {\(y\)};
			
			        % draw the unit circle
			        \draw[thick] (0cm,0cm) circle(1cm);
			
			        \foreach \x in {0,30,...,360} {
			                % lines from center to point
			                \draw[gray] (0cm,0cm) -- (\x:1cm);
			                % dots at each point
			                \filldraw[black] (\x:1cm) circle(0.4pt);
			                % draw each angle in degrees
					\fill[white] (\x:0.4cm) circle (0.12);
			        }
			        
			        \foreach \x in {45,135,225,315} {
			        	% lines from center to point
			                \draw[gray] (0cm,0cm) -- (\x:1cm);
			                % dots at each point
			                \filldraw[black] (\x:1cm) circle(0.4pt);
			                % draw each angle in degrees
					\fill[white] (\x:0.4cm) circle (0.12);
			        }
			
			        % draw each angle in radians
			        \foreach \x/\xtext in {
			            30/\frac{\pi}{6},
			            45/\frac{\pi}{4},
			            60/\frac{\pi}{3},
			            90/\frac{\pi}{2},
			            120/\frac{2\pi}{3},
			            135/\frac{3\pi}{4},
			            150/\frac{5\pi}{6},
			            180/\pi,
			            210/\frac{7\pi}{6},
			            225/\frac{5\pi}{4},
			            240/\frac{4\pi}{3},
			            270/\frac{3\pi}{2},
			            300/\frac{5\pi}{3},
			            315/\frac{7\pi}{4},
			            330/\frac{11\pi}{6},
			            360/2\pi}
			                \fill[white] (\x:0.8cm) circle (0.12);

				\foreach \x/\xtext/\y in {
			            % the coordinates for the first quadrant
	    			    30/\frac{\sqrt{3}}{2}/\frac{1}{2},
			     		45/\frac{\sqrt{2}}{2}/\frac{\sqrt{2}}{2},
			            60/\frac{1}{2}/\frac{\sqrt{3}}{2},
			            % the coordinates for the second quadrant
			            150/-\frac{\sqrt{3}}{2}/\frac{1}{2},
			            135/-\frac{\sqrt{2}}{2}/\frac{\sqrt{2}}{2},
			            120/-\frac{1}{2}/\frac{\sqrt{3}}{2},
			            % the coordinates for the third quadrant
			            210/-\frac{\sqrt{3}}{2}/-\frac{1}{2},
			            225/-\frac{\sqrt{2}}{2}/-\frac{\sqrt{2}}{2},
			            240/-\frac{1}{2}/-\frac{\sqrt{3}}{2},
			            % the coordinates for the fourth quadrant
			            330/\frac{\sqrt{3}}{2}/-\frac{1}{2},
			            315/\frac{\sqrt{2}}{2}/-\frac{\sqrt{2}}{2},
			            300/\frac{1}{2}/-\frac{\sqrt{3}}{2}}
			                \draw (\x:1.25cm) node[fill=white] {};
			
			        % draw the horizontal and vertical coordinates
			        % the placement is better this way
			        \draw (-1.25cm,0cm) node[above=1pt] {$(-1,0)$}
			              (1.25cm,0cm)  node[above=1pt] {$(1,0)$}
			              (0cm,-1.25cm) node[fill=white] {$(0,-1)$}
			              (0cm,1.25cm)  node[fill=white] {$(0,1)$};
	   	 \end{tikzpicture}
	    	\newpage
	    	
		\begin{rmk}[Common Trig Identities]
			\\[75pt]
		\end{rmk}
		\begin{ex}
			Sketch the graph of each trig function over one period.  If necessary, sketch any asymptotes the graph has.
		\end{ex}
			\vs{2}
			
		\begin{ex}
			Find the domain of the function \(f(x) = \dfrac{3}{2\sin x + 1}\), first on the interval \([0,2\pi)\), then in general.
		\end{ex}
			\vs{1}
			\newpage

	\subsection*{Algebra}
		\begin{rmk}[Average Rate of Change/Slope]
			Given two points \((x_1,y_1)\) and \((x_2,y_2)\), where \(y_i = f(x_i)\), the slope of the line between the points (also called the average rate of change between the two points) is given by
			\\[50pt]
		\end{rmk}
		
		\begin{rmk}[Forms of Lines]
			Consider a line passing through the point \((x_1,y_1)\) with slope \(m\). In \textbf{point-slope form}, the line can be written as\\[75pt]
			
			If the line has a known \(y-\)intercept at \((0,b)\), then in \textbf{slope-intercept form}, the line can be written as\\[50pt]
		\end{rmk}
		
		\begin{ex}
			Consider the points \((11,-4)\) and \((-4,5)\). 
			\begin{enumerate}[(a)]
				\item Find the slope of the line between the two points.
					\vs{1}
					
				\item Write the line in point-slope and slope-intercept forms.
					\vs{1}					
			\end{enumerate}
		\end{ex}
			\newpage
			
		
		\begin{rmk}[Quadratic Formula]
			For any polynomial of the form \(f(x) = ax^2 + bx + c\), the roots/zeros/\(x-\)intercepts of \(f\) are located at\\[75pt]
		\end{rmk}
		
		\begin{ex}
			For each polynomial, describe the end behavior of \(f(x)\) as \(x\to \pm \infty\), find all zeros of \(f\), and sketch a graph of \(f\).
			\begin{enumerate}[(a)]
				\item \(f(x) = -2x^2+4x-1\)
					\vs{1}
					
				\item \(f(x) = x^3-3x^2-4x\)
					\vs{1}
			\end{enumerate}
		\end{ex}	
			\newpage
			